


%%%%%%%%%%%%%%%%%%%%
% Data
\section{ecPoint-Rainfall}




%%%%%%%%%%%%%%%%%%%%
% Results
\section{Results}
\subsection{Hungary}
\subsubsection{Developed products}
\subsubsection{Independent verification}
\subsection{Costa Rica}
\subsubsection{Developed products}
\subsubsection{Independent verification}


%%%%%%%%%%%%%%%%%%%%
% Discussions
\section{Discussions}


\subsection{Forecasters' dilemma, torn between maintaining or modernizing their working practices}
Some meteorological centres, especially those with less experience with probabilistic products seem to be more inclined to use probabilistic products to formulate their forecasts, but to not communicate their forecasts in a probabilistic way to the public: \textit{Don't air your "uncertainty" in public.} \citep{jimenez2020}

\subsection{Understanding of probabilistic forecasts}
Whether we are talking of expert meteorologists \citep{Stewart2016} or the general public \citep{Gigerenzer2005}, understanding correctly probabilistic weather forecasts is not an easy task.

\subsection{Adjustments to objective ensemble guidance}
\citet{Novak2008} has shown how 

\subsection{Smaller or higher percentiles, this is the dilemma!}
\subsection{Practice makes perfect}